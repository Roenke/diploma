В настоящее время объектно-ориентированные языки программирования заимствуют некоторые возможности функциональных языков.  Обычно, это позволяет расширить синтаксис языка, упростить написание некоторых операций, а так же сделать код более выразительным. Одним из основных понятий функциональных языков является функция высшего порядка -- функция, принимающая в качестве аргумента другие функции или возвращающая их в качестве результата своей работы. Примерами языков, активно использующих функции высшего порядка являются C\#, Scala, Kotlin и другие. В восьмой версии Java появилась возможность использовать анонимные функции и стандартная библиотека была расширена пакетом java.util.stream, содержащим набор классов и интерфейсов, позволяющий обрабатывать потоки объектов при помощи функций высшего порядка.

Отладка программы - это один из этапов разработки программного обеспечения, в ходе которого разработчик находит и исправляет ошибки. Отладка может производиться с помощью нескольких подходов. Один их таких подходов -- использование отладчика -- отдельной программы, позволяющей управлять процессом исполнения другой программы, просматривать и модифицировать её внутреннее состояние.

В случае java добавление анонимных функций привело к созданию нового способа написания кода для обработки элементов коллекций, TODO. К плюсам использования Stream API

В рамках данной работы будет разработано расширение для среды разработки IntelliJ IDEA, позволяющее упростить процесс отладки кода с использованием функций высшего порядка. В главе \ref{chapter1} представлен обзор имеющихся возможностей по отладке кода на языке Java, разобраны особенности использования классов из пакета java.util.stream, а так же перечислены трудности, возникающие при отладке программ, которые используют эти классы. Так же в первой главе рассмотрены попытки решения проблемы для других языков. В главе \ref{chapter2} приводится обоснование предположений сделанных о коде пользователя, описывается основная идеи и алгоритмы, которые были использованы при решении задачи. В главе \ref{chapter3} перечисляются особенности реализации, используемые возможности среды разработки, объясняются принятые архитектурные решения, обзор пользовательского интерфейса.

