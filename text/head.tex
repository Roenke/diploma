% Год, город, название университета и факультета предопределены,
% но можно и поменять.
% Если англоязычная титульная страница не нужна, то ее можно просто удалить.
\filltitle{ru}{
    chair              = {Кафедра математических и информационных технологий},
    title              = {Отладка операций в функциональном стиле на языке Java в среде разработки IntelliJ IDEA},
    % Здесь указывается тип работы. Возможные значения:
    %   coursework - Курсовая работа
    %   diploma - Диплом специалиста
    %   master - Диплом магистра
    %   bachelor - Диплом бакалавра
    type               = {master},
    position           = {студента},
    group              = 604,
    author             = {Бибаев Виталий Игоревич},
    supervisorPosition = {ООО ИнтеллиДжей Лабс, разработчик платформы IntellJ },
    supervisor         = {Ушаков Е.\,А.},
    reviewerPosition   = {ООО ИнтеллиДжей Лабс, разработчик Scala plugin},
    reviewer           = {Тропин Н.\,В.},
    chairHeadPosition  = {д.\,ф.-м.\,н., профессор},
    chairHead          = {Омельченко А.\,В.},
    % university = {САНКТ-ПЕТЕРБУРГСКИЙ АКАДЕМИЧЕСКИЙ УНИВЕРСИТЕТ},
    % faculty = {Центр высшего образования},
    % city = {Санкт-Петербург},
    % year             = {2013}
}
\filltitle{en}{
    chair              = {Department of Mathematics and Information Technology},
    title              = {Debugging of functional-style operations in Java in the IntelliJ IDEA IDE},
    author             = {Vitaliy Bibaev},
    %TODO: Intellij labs
    supervisorPosition = {Software Developer in IntelliJ Platform at IntelliJ labs},
    supervisor         = {Egor Ushakov},
    reviewerPosition   = {Software Developer in Scala plugin at IntelliJ labs},
    reviewer           = {Nikolay Tropin},
    chairHeadPosition  = {professor},
    chairHead          = {Alexander Omelchenko},
}
\maketitle
\tableofcontents