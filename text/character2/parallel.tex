Потоки объектов могут использовать параллелизм для оптимизации производительности. Для того, чтобы некоторый вызов исполнялся с использованием нескольких потоков, достаточно вызвать метод parallelStream у источника, либо метод parallel у уже имеющегося объекта Stream. Пример: 
\inputminted{java}{chapter2/code/ParallelStream.java}
При запуске вычислений в параллельном потоке, исходный поток разбивается на множество мелких задач, которые выполняются при помощи Fork/Join Pool \cite{java:forkjoin}, работающий по принципу work-stealing \cite{wiki:worksteal}. При этом гарантии могут нарушаться -- промежуточные операции могут быть уже не всегда ленивыми. Это связано с желанием уменьшить необходимую синхронизацию между потоками, поэтому могут выполняться избыточные действия.

Мы не будем ставить задачу научиться отлаживать параллельные потоки. Как правило, там возникают совершенно другие проблемы и для их решения используются подходы, отличные от использования отладчика. Основная цель параллельных потоков заключается в том, чтобы логика, работающая корректно последовательно, работала быстрее. Поэтому разумно оставить возможность отлаживать такие вызовы как последовательные.