Настоящая работа посвящена проблемам отладки кода, использующего пакет \\ java.util.stream стандартной библиотеки языка java. В работе исследованы имеющиеся способы нахождения ошибок в таком коде, а так же описаны их недостатки. Были сформулированы требования к решению, которое упростит процесс отладки. Был описан подход, который позволит получить информацию о последовательности исполнения вызова Stream API и данных о трансформациях каждого объекта внутри этого вызова. Описанный подход был реализован в качестве расширения для среды разработки IntelliJ IDEA, упрощающего отладку вызовов Stream API. Расширение доступно для пользователей \cite{java:stream-debugger}.

В качестве возможных направлений развития следует отметить:
\begin{itemize}
	\item Поддержка операций над потоками элементов, предоставляемых библиотекамиы StreamEx и jOOL.
	\item Обобщение решения для возможности отладки функциональных операций в других языках -- Scala, Kotlin, C\#.
	\item Исследование возможности сбора информации об объектах внутри потока при помощи точек останова.
\end{itemize}