\section{Подход к решению задачи}
Прежде чем приступить к описанию решения, необходимо понять, какие гарантии предоставляет Stream API, а так же какие предположения можно делать о пользовательском коде.

\subsection{Гарантии Stream API}\label{java:guarantees}
Промежуточные операции всегда ленивые. Это значит, что операция не будет запрашивать следующий элемент, если может корректно вычислиться без его использования.
\begin{itemize}
	\item Это исключает ситуации, когда из каких-то промежуточных операций вынимались объекты, но затем не использовались.
	\item Но это не исключает, что некоторым промежуточным операциям потребуется прочитать более одного объекта, чтобы что-либо вычислить. (sorted, distinct)
	\item Следствие: нет вызова терминальной операции - нет вычислений.
\end{itemize}

\subsection{Требования к коду пользователя} \label{code:demands}

Кроме предоставления гарантий, Stream API делает предположения о коде пользователя. Соблюдение следующих требований снизит вероятность неожиданного поведения вызовов Stream API. Эти требования особенно критичны в случае использования параллельных потоков.

\begin{itemize}
	\item Операции над объектами не могут модифицировать объект-источник.
	\inputminted{java}{chapter2/code/SourceModification.java}
	\item Функции над объектами в потоке не должны иметь состояния (когда это возможно).
	\inputminted{java}{chapter2/code/StatefulOperation.java}
	\item Однажды созданный объект Stream может вызвать лишь одну терминальную операцию. Если это требование нарушено, то произойдет исключение.
	\inputminted{java}{chapter2/code/OneTermination.java}
\end{itemize}

\subsection{Параллельные потоки объектов}
Потоки объектов могут использовать параллелизм для оптимизации производительности. Для того, чтобы некоторый вызов исполнялся с использованием нескольких потоков, достаточно вызвать метод parallelStream у источника, либо метод \mintinline{java}{parallel} у уже имеющегося объекта \mintinline{java}{Stream}. Пример: 
\inputminted{java}{chapter2/code/ParallelStream.java}
При запуске вычислений в параллельном потоке, исходный поток разбивается на множество мелких задач, которые выполняются при помощи\textit{ Fork/Join Pool} \cite{java:forkjoin}, работающий по принципу \textit{work-stealing} \cite{wiki:worksteal}. При этом гарантии могут нарушаться -- промежуточные операции могут быть уже не всегда ленивыми. Это связано с желанием уменьшить необходимую синхронизацию между потоками, поэтому могут выполняться избыточные действия.

Мы не будем ставить задачу научиться отлаживать параллельные потоки. Как правило, там возникают совершенно другие проблемы и для их решения используются подходы, отличные от использования отладчика. Основная цель параллельных потоков заключается в том, чтобы логика, работающая корректно последовательно, работала быстрее. Поэтому разумно оставить возможность отлаживать такие вызовы как последовательные.

Для того, чтобы избавиться от параллелизма можно после каждого вызова, который делает поток параллельным добавить вызов \mintinline{java}{sequential}, тогда поток будет всюду последовательным. Но в текущей реализации поток не может одновременно содержать однопоточные и многопоточные части, поэтому, чтобы сделать его полностью последовательным будет достаточно добавить вызов \mintinline{java}{sequential} перед завершающим вызовом. Это обусловлено дизайном библиотеки, т.к. завершающая операция запускает все вычисления, то и она "принимает решение" сколько потоков использовать.


\subsection{Определение подходящего вызова}
При отладке программ внутри среды разработки пользователь видит текущее положение программы относительно исходного кода и текущее состояние программы (локальные переменные, объекты и их поля, потоки, стек и другие). На самом деле виртуальная машина Java исполняет инструкции байт-кода и какие они в точности нам неизвестно. Можно считать, что есть какой-то способ их отображение на исходный код, который реализует среда разработки, благодаря которому мы можем говорить о текущее положении программы не в термах машинных комманд или java байт-кода, а в терминах исходного java кода. Поэтому далее под позицией отладчика будем понимать строку в исходном коде.

Прежде чем начать отладку цепочки Stream API необходимо определить её границы и понять достаточно ли данных для её вычисления.

В отладчике для каждого из потоков всегда определено текущее положение указателя инструкций. Определим подходящие положения этого указателя относительно вызова цепочки Stream API для того, чтобы начать её отладку. При вызове цепочки в ней могут участвовать объявленные ранее переменные и значения полей. Поэтому отлаживать вызов, который находится значительно позже чем текущая инструкция не имеет смысла (возможно, имеющихся данных будет недостаточно). Текущая строка отладчика отмечена стрелкой (=>).

\inputminted{java}{chapter2/code/FarToCall.java}

Если же все инструкции до вызова выполнены, значит все переменные и поля были инициализированы и можно начать его отладку. 

\inputminted{java}{chapter2/code/BeforeCall.java}

Начинать отладку можно и после начала, т.к. мы предположили, что вызов не имеет побочных эффектов, которые изменят окружение.
\inputminted{java}{chapter2/code/InEvaluation.java}

После того, как вызов завершен, он больше не подходит для отладки. Во-первых, может быть уже другой вызов, который подходит. Во-вторых окружение может измениться, изменив начальную семантику вызова.
\inputminted{java}{chapter2/code/AfterCall.java}

Таким образом, отладка допустима, если текущее положение отладчика между непосредственным начало вызова потока и до его завершения.

\subsubsection{Цепочки в других элементах стека вызовов}
При отладке пользователь может сменить элемент стека вызовов. В другом элементе стека позиция отладчика находится на вызове метода. Этот вызов может находиться внутри цепочки Stream API. А значит, такая цепочка подходит для отладки.
\inputminted{java}{chapter2/code/NestedMethod.java}

При переключении элемента стека в функцию \textbf{nestedExample} вызов Stream API внутри неё должен быть доступен для отладки.

\subsubsection{Неоднозначные вызовы}
Из одного и того же положения отладчика может быть доступно сразу несколько подходящих вызовов Stream API:
\begin{itemize}
	\item Внутри одного арифметического выражения;
	\inputminted{java}{chapter2/code/AmbiguousPlus.java}
	\item Аргументы некоторого вызова;
	\inputminted{java}{chapter2/code/AmbiguousArgs.java}
	\item Параметр другого вызова Stream API;
	\inputminted{java}{chapter2/code/AmbiguousNested.java}
	\item Вызов находится в объемлющем коде;
	\inputminted{java}{chapter2/code/AmbiguousLambda.java}
	\item В одной цепочке методов есть несколько последовательных цепочек с терминальной операцией;
	\inputminted{java}{chapter2/code/AmbiguousLinkedChains.java}
\end{itemize}
Все такие цепочки подходят для отладки.

\subsubsection{Незавершенные цепочки}\label{java:incomplete}
Согласно \ref{java:guarantees} цепочки методов Stream API без терминальной операции не вычисляются, значит и отлаживать их не нужно.

\subsubsection{Присваивание объекта потока в переменную}
Поскольку Stream API это просто набор классов, а не специальный синтаксис, то цепочки создают обычные объекты, с которыми можно обращаться как с любыми другими объектами. В этом контексте интерес представляет операция присваивания в переменную.
\inputminted{java}{chapter2/code/AssignToVariable.java}

Это вносит некоторые проблемы. Ранее мы рассматривали цепочке методов Stream API как нечто неделимое, но теперь видим, что это не так. Причем выражение, которое инициализирует переменную \mintinline{java}{even} без завершающей операции, а значит, это выражение нельзя отлаживать (см \ref{java:incomplete}).

С другой стороны, выражение \mintinline{java}{even.count()} нельзя запустить повторно.

Этот случай достаточно прост, и можно решить, что вы можем посмотреть как инициализируется переменная \mintinline{java}{even}. После чего объединить это выражение с терминальной операцией, получив 

\mintinline{java}{final long size = collection.stream().filter(x -> x % 2 == 0).count()}
	
Но это не решит всех наших проблем. Этот пример слишком прост, и на практике могут встретиться и более сложные случаи:
\inputminted{java}{chapter2/code/AssignToVariableHardCase.java}

В этом случае тем же способом уже не справиться (хотя этот случай выглядит более близким к практике). Поэтому для таких вызовов будем находить лишь ту часть, которая относится только к последней цепочке (с терминальной операцией). Заметим, что это инвалидирует поток в переменной even, поэтому об этому лучше предупреждать пользователя.

\subsection{Построение состояний и переходов}
В \ref{code:demands} описаны требования к коду пользователя, который использует Stream API. Эти требования позволяют сделать предположение, что вызов не имеет побочных эффектов и его можно повторить несколько раз, получив тот же самый результат.

Основная идея состоит в том, чтобы запустить вычисление модифицированного вызова Stream API, результат которого совпадет с исходным, при этом новая цепочка собирает информацию, полезную для процесса отладки.

\subsubsection{Выражение для сбора отладочной информации}\label{build-expression}
Для того, чтобы собрать информацию о том, какие объекты проходили через поток можно использовать метод \mintinline{java}{peek}. Добавив такой вызов между каждой промежуточной операцией мы сможем найти какие объекты были внутри потока.

Рассмотрим пример:

Данный вызов возвращает список имен людей, чей возраст меньше 18 лет, упорядоченный по возрастанию.
\inputminted{java}{chapter2/code/StreaMWithoutPeeks.java}
При помощи метода \mintinline{java}{peek} можно наблюдать промежуточные состояния в потоке:
\inputminted{java}{chapter2/code/StreamWithPeeks.java}

Даже если мы будем просто печатать элементы в методе peek (с указанием где именно, то мы получим нечто полезное). Для некоторого списка людей вызов может напечатать следующее. В квадратных скобках указан возраст. 
\inputminted{java}{chapter2/code/peekResults.txt}

Заметим, что из такого вывода можно извлечь следующую информацию:
\begin{itemize}
	\item \textbf{Последовательность} прохождения объектов через цепочку вызовов. Напечатанные строки идут в порядке исполнения программы. Значит, сначала из источника был взят объект с \mintinline{java}{id = 1}, затем он прошел фильтрацию и попал в sorted, и так далее.
	\item \textbf{Результат фильтрации}. Можно увидеть все значения, которые прошли фильтрацию -- это ровно те значения, которые мы наблюдали после вызова \mintinline{java}{filter}.
	\item \textbf{Свойства вызовов.} Вызов sorted имеет состояние и требует выполнить весь поток до своего вызова. Поэтому пока объекты в источнике не закончились после вызова sorted ничего не происходило.
	\item \textbf{Множества } объектов до и после вызова. Обратим внимание, что мы печатаем положение метода peek, поэтому можно понять множества объектов, которые были до и после каждого из вызовов.
	\item \textbf{Преобразования} объектов. На примере вызова map, видно, что в последовательных событиях содержится преобразование -- извлечение имени человека: \mintinline{java}{{id = 1, name = Vasily, age = 10} => Vasily}
\end{itemize}

Приведенный лог выше плох тем, что он описывает поведение всего потока целикомЭ, а не каждого вызова в отдельности. Но для каждого вызова в цепочки можно оставить только зависи, которые были сделаны непосредственно до и после него. Таким образом получим два набора объектов -- до вызова и после него. Объекты в этих наборах можно упорядочить по времени появления в логе. 

То есть для каждого промежуточного вызова можем добавить 2 метода peek, которые соберут информацию об объектах, которые были до него и после него. Кроме того, добавим глобальное время, которое будет увеличиваться когда следующий элемент был запрошен у одного из исходных промежуточных объектов Stream.

\inputminted{java}{chapter2/code/LocalChainModification.java}

Таким образом, для вызова \mintinline{java}{call} будет два множества пар: 
\begin{equation*}
	L = \{t_i, obj_i\}, R = \{t_j, obj_j\}
\end{equation*}

Объекты полученные до вызова \mintinline{java}{call} обозначим $L$, а после \mintinline{java}{call} -- $R$

После этого нужно восстановить переходы между множествами $L$ и $R$, которые соответствуют вызову \mintinline{java}{call}.

\textbf{Прямым переходом} для элемента $l \in L$ множество $R' \subset R$ такое, что $\forall r \in R'$ объект $r$ является результатом воздействия операции \mintinline{java}{call} на элемент $l$.

По аналогии можно определить и \textbf{обратный переход}.
Правила, по которым строятся прямые и обратные отображения для конкретных вызовов описаны в \ref{interpret}.

Для большинства промежуточных операций множеств $L$ и $R$ достаточно для построения переходов. Единственное исключение -- промежуточная операция \mintinline{java}{distinct}. Решение для него предложим немного позже в TODO;

\subsubsection{Вычисление выражения}
Все отладчики внутри сред разработки предоставляют возможность вычислять выражения на java код при попадании на точку останова.

\subsubsection{Интерпретация результата}\label{interpret}

Как мы сказали в \ref{build-expression} результатом выполнения выражения является набор множеств $L$ и $R$ для каждой промежуточной операции. Рассмотрим алгоритмы построения переходов для каждого из промежуточных вызовов стандартной реализации Stream API.
\begin{itemize}
	\item map -- 
	\item filter --
	\item flatMap --
	\item sorted --
\end{itemize}

Выше перечислены методы для которых построение переходов различается. Для остальных вызовов можно использовать эти же алгоритмы, т.к. они в смысле переходов являются частным случаев описанных выше операций.
\begin{itemize}
	%linked
	\item limit --
	\item skip --
	\item peek --
	\item onClose -- 
\end{itemize}  


\subsubsection{Решение для операции distinct}
$distinct$ -- промежуточная операция c состоянием, поэтому прежде чем что-то вернуть ей может понадобиться прочитать весь входной поток. Результатом является поток объектов, для которого гарантируется, что все объекты попарно различны (в смысле equals\cite{java:equals}). Новых объектов при этом появиться, очевидно не может.

Условие что операция имеет состояние не позволяет использовать времена $t_i, t_j$ для разрешения порядка. Поэтому мы может только сравнивать объекты. 

\subsection{Визуализация}

