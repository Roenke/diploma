\section{Подход к решению задачи}
Прежде чем приступить к описанию решения, необходимо понять, какие гарантии предоставляет Stream API, а так же какие предположения можно делать о пользовательском коде.

\subsection{Гарантии Stream API}
Промежуточные операции всегда ленивые. Это значит, что операция не будет запрашивать следующий элемент, если может корректно вычислиться без его использования.
\begin{itemize}
	\item Это исключает ситуации, когда из каких-то промежуточных операций вынимались объекты, но затем не использовались.
	\item Но это не исключает, что некоторым промежуточным операциям потребуется прочитать более одного объекта, чтобы что-либо вычислить. (sorted, distinct)
	\item Следствие: нет вызова терминальной операции - нет вычислений.
\end{itemize}

\subsection{Требования к коду пользователя}

Кроме предоставления гарантий, Stream API делает предположения о коде пользователя. Соблюдение следующих требований снизит вероятность неожиданного поведения вызовов Stream API. Эти требования особенно критичны в случае использования параллельных потоков.

\begin{itemize}
	\item Операции над объектами не могут модифицировать объект-источник.
	\inputminted{java}{chapter2/code/SourceModification.java}
	\item Функции над объектами в потоке не должны иметь состояния (когда это возможно).
	\inputminted{java}{chapter2/code/StatefulOperation.java}
	\item Однажды созданный объект Stream может вызвать лишь одну терминальную операцию. Если это требование нарушено, то произойдет исключение.
	\inputminted{java}{chapter2/code/OneTermination.java}
\end{itemize}

\subsection{Параллельные потоки объектов}
Потоки объектов могут использовать параллелизм для оптимизации производительности. Для того, чтобы некоторый вызов исполнялся с использованием нескольких потоков, достаточно вызвать метод parallelStream у источника, либо метод \mintinline{java}{parallel} у уже имеющегося объекта \mintinline{java}{Stream}. Пример: 
\inputminted{java}{chapter2/code/ParallelStream.java}
При запуске вычислений в параллельном потоке, исходный поток разбивается на множество мелких задач, которые выполняются при помощи\textit{ Fork/Join Pool} \cite{java:forkjoin}, работающий по принципу \textit{work-stealing} \cite{wiki:worksteal}. При этом гарантии могут нарушаться -- промежуточные операции могут быть уже не всегда ленивыми. Это связано с желанием уменьшить необходимую синхронизацию между потоками, поэтому могут выполняться избыточные действия.

Мы не будем ставить задачу научиться отлаживать параллельные потоки. Как правило, там возникают совершенно другие проблемы и для их решения используются подходы, отличные от использования отладчика. Основная цель параллельных потоков заключается в том, чтобы логика, работающая корректно последовательно, работала быстрее. Поэтому разумно оставить возможность отлаживать такие вызовы как последовательные.

Для того, чтобы избавиться от параллелизма можно после каждого вызова, который делает поток параллельным добавить вызов \mintinline{java}{sequential}, тогда поток будет всюду последовательным. Но в текущей реализации поток не может одновременно содержать однопоточные и многопоточные части, поэтому, чтобы сделать его полностью последовательным будет достаточно добавить вызов \mintinline{java}{sequential} перед завершающим вызовом. Это обусловлено дизайном библиотеки, т.к. завершающая операция запускает все вычисления, то и она "принимает решение" сколько потоков использовать.


\subsection{Поиск подходящего вызова}

\subsection{Построение выражения для вычисления}

\subsection{Вычисление выражения}

\subsection{Интерпретация результата}

\subsection{Визуализация}
