
%\input{frames/overview.tex}
\section{Введение}
\subsection{Язык программирования Java}
\begin{frame}
	\frametitle{\insertsection} 
	\framesubtitle{\insertsubsection}
	\textit{Java} - строго типизированный объектно-ориентированный язык программирования, разработанный 
	компанией Sun Microsystems (приобретена Oracle). Приложения Java обычно транслируются в специальный байт-код, поэтому 
	они могут работать на любой компьютерной архитектуре, с помощью \textbf{виртуальной Java-машины}.
\end{frame}



% Streams
\subsection{Интерфейс потоков объектов}
\begin{frame}
\frametitle{Java Stream API} % Table of contents slide, comment this block out to remove it
% Фича Java 8
% Ленивость
% Функциональный стиль
\begin{itemize}
	\item Stream API - пакет java.util.stream, содержащий набор классов и интерфейсов для упрощения обработки последовательностей элементов.
	\item Появилось в Java 8 
	\item Аналоги в других языках - LINQ в C\#
\end{itemize}
\end{frame}
\subsection{Интерфейс потоков объектов. Пример.}
\begin{frame}[fragile]
\frametitle{\insertsection} 
\framesubtitle{\insertsubsection}
\inputminted{java}{code/ExampleClass.java}
\end{frame}
\begin{frame}
	\frametitle{\insertsection} 
	\framesubtitle{Задача}
	Найти имена людей, чей возраст меньше 18 лет, отсортировав по возрасту
\end{frame}
\begin{frame}
\inputminted{java}{code/CyclesUsage.java}
\end{frame}
\begin{frame}
\inputminted{java}{code/CollectionUsage.java}
\end{frame}
\begin{frame}
\inputminted{java}{code/StreamUsage.java}
\end{frame}
\subsection{Интерфейс потоков объектов. Реализация}
\begin{frame}
\frametitle{\insertsection} 
\framesubtitle{\insertsubsection}
\begin{itemize}
	\item \textbf{Источник объектов}. Например, массив, коллекция, функция генератор, поток ввода и другие.
	\inputminted{java}{code/Producer.java}
	\item \textbf{Промежуточные операции}. Преобразуют объекты внутри потока.
	\begin{itemize}
		\item Без состояния
		\inputminted{java}{code/Stateless.java}
		\item С состоянием
		\inputminted{java}{code/Stateful.java}
	\end{itemize}
	\item \textbf{Завершающая операция}. Завершает цепочку, преобразуя поток объектов в результат.
	\inputminted{java}{code/Termination.java}
\end{itemize}
\end{frame}

\begin{frame}
\frametitle{Флаги операций} % Table of contents slide, comment this block out
\begin{itemize}
	\item Distinct - все объекты в стриме различны (относительно equals)
	\item Sorted - объекты упорядочены (естественный порядок для Comparable объектов)
	\item Sized - известно количество объектов
	\item Short circuit - вызов может завершиться не просматривая все объекты
\end{itemize}
\end{frame}
\begin{frame}
\frametitle{\insertsection} 
\framesubtitle{\insertsubsection}
\begin{itemize}
	
	\item Существуют расширения $java.util.stream$ - StreamEx, jOOL. Эти библиотеки совместимы со стандартной реализацией, расширяя стандартные интерфейсы.
\end{itemize}
\end{frame}


% Debugger
\subsection{Отладчик}
\begin{frame}
\frametitle{Отладка Java программ} % Table of contents slide, comment this block out to remove it
TODO: Развернуто перечислить основные функции
\end{frame}
\begin{frame}
\frametitle{\insertsection} 
\framesubtitle{\insertsubsection}
\begin{itemize}
	\item JDB (java debugger) - простой command-line отладчик для java. Поставляется с JDK. Предоставляет примитивные операции:
	\begin{itemize}
		\item Подключиться к виртуальной машине (возможно, удаленной)
		\item Добавить точку останова
		\item Показать список потоков
		\item ...
	\end{itemize}
	jdb является достаточно примитивной демонстрацией возможностей JDPA (Java Platform Debugger Architecture)
	\item Отладчики внутри среды разработки - IntelliJ IDEA, Eclipse, NetBeans и т.д. Они так же 
	используют JDPA, но активнее чем JDB
\end{itemize}
\end{frame}

\begin{frame}
\frametitle{\insertsection} 
\framesubtitle{\insertsubsection}
JDPA - инфраструктура для отладки Java приложения. Содержит набор интерфейсов и протоколов:
\begin{itemize}
	\item JVM TI - Java VM Tool Interface - описывает сервисы, которая предоставляет виртуальная машина для отладки. (TODO: сказать про примеры)
	\item JDWP - Java Debug Wire Protocol - протокол общения между отладчиком и виртуальной машиной. (TODO: сказать про абстракцию)
	\item JDI - Java Debug Interface - высокоуровневая Java библиотека для взаимодействия с виртуальной машиной. (сказать про зеркала реальных объектов)
\end{itemize}
\end{frame}

\begin{frame}
\frametitle{Java Debug Interface} % Table of contents slide, comment this block out to remove it
TODO: JVMTI и это вот всё
\end{frame}

% Stream debugging
\section{Отладка потоков объектов}
\subsection{Поддержка со стороны сред разработки}
\begin{frame}
\frametitle{Сравнение отладки императивного кода и Stream API} % Table of contents slide, comment this block out to remove it
TODO: Построить таблицу, показать как можно, и как нет в обоих случаях
\end{frame}

\subsection{Проблемы}
\begin{frame}
\frametitle{\insertsection} 
\framesubtitle{\insertsubsection}
Недостатки отладки интерфейса потоков объекта в сравнении с обычными управляющими структурами
\begin{itemize}
	\item Нетривиальная последовательность исполнения
	\item Отсутствие знаний о промежуточных результатах вычислений 
	\item Отсутствие информации о трансформации объектов
	\item Сложные стеки вызовов
	\item Отладка исключительных ситуаций
\end{itemize}
\end{frame}

\subsection{Как сейчас можно отладить поток объектов}
\begin{frame}
\frametitle{\insertsection} 
\framesubtitle{\insertsubsection}
\begin{itemize}
	\item Точки останова (с условием)
	\item Последовательное исполнение
	\item При помощи метода peek
	\item Автоматическая конвертация в код с обычными управляющими структурами и отладка этого кода
	\item Частичное исполнение и сохранение во временную коллекцию (не во всех средах разработки)
\end{itemize}
\end{frame}

\subsection{Пример использования интерфейса потоков в проекте с открытым кодом}
\begin{frame}
\frametitle{\insertsection} 
\framesubtitle{\insertsubsection}
\inputminted{java}{code/Hard.java}
\end{frame}


\section{Цель}
\begin{frame}
	\frametitle{\insertsection} 
	\framesubtitle{\insertsubsection}
	Расширить возможности отладчика для поиска ошибок при использовании интерфейса потоков объектов
\end{frame}

\subsection{Существующие решения}
\begin{frame}
\frametitle{\insertsection} 
\framesubtitle{\insertsubsection}
Существует расширение для отладчика Visual Studio для C\# - OzCode. Одна из его функций - отладчик для LINQ.
\includegraphics[scale=0.35]{img/ozcode.png}
\end{frame}

\subsection{Особенности OzCode}
\begin{frame}[noframenumbering]
\frametitle{\insertsection} 
\framesubtitle{\insertsubsection}
\includegraphics[scale=0.35]{img/ozcode.png}
\end{frame}


\section{Решение}
\subsection{Гарантии, предоставляемые библиотекой}
\begin{frame}[noframenumbering]
\frametitle{\insertsection} 
\framesubtitle{\insertsubsection}
Поток объектов всегда ленивый. Это значит, что объекты не из источника будут браться только когда выполняется терминальная операция и ей нужен объект.
\begin{itemize}
	\item Это исключает ситуации, когда из каких-то промежуточных операций требовались объекты, но затем не использовались
	\item Но это не исключает, что некоторым промежуточным операциям потребуется прочитать более одного объекта, чтобы вернуть один. (см sorted, distinct, и др)
	\item Следствие: нет вызова терминальной операции - нет вычислений
\end{itemize}

\end{frame}
\subsection{Требования к пользовательскому коду}
\begin{frame}
\frametitle{\insertsection} 
\framesubtitle{\insertsubsection}
\begin{itemize}
	\item Операции над объектами не могут модифицировать объект - источник (если он есть)
	\item Функции над объектами в потоке не должны иметь состояния (в большинстве случаев)
	\item Однажды созданный объект потока объектов может вызвать лишь одну терминальную операцию.
	\inputminted{java}{code/OneTermination.java}
\end{itemize}
\end{frame}
\subsection{Ограничения и особенности}
\begin{frame}
\frametitle{Повторное использование объектов Stream} % Table of contents slide, comment this block out
Однажды созданный объект Stream может вызвать лишь одну терминальную операцию

TODO: вставить код в котором есть такая ошибка.

(В C\# это допускается)

TODO: код из C\#
\end{frame}
\begin{frame}
\frametitle{\insertsection} 
\framesubtitle{\insertsubsection}
Потоки объектов можно присваивать в промежуточные переменные.

\inputminted{java}{code/VariableAssign.java}
\end{frame}

\subsection{Метод peek}
\begin{frame}[noframenumbering]
\frametitle{\insertsection} 
\framesubtitle{\insertsubsection}
Интерфейс Stream определяет для отладочных целей метод peek. Добавим его между всеми вызовами методов Stream<T>.
\inputminted{java}{code/StreamWithPeeks.java}
\end{frame}

\begin{frame}[noframenumbering]
\frametitle{\insertsection} 
\framesubtitle{\insertsubsection}
\fboxsep=0pt
\noindent
	\begin{minipage}[t]{0.48\linewidth}
		Информация, которую можно извлечь:
		\begin{itemize}
			\item Последовательность прохождения объектов через цепочку вызовов
			\item Результат фильтрации
			\item sorted имеет состояние и требует выполнить весь поток до своего вызова
			\item Преобразования объектов
		\end{itemize}
	\end{minipage}
	\hfill%
		\begin{minipage}[t]{0.48\linewidth}
			Результат вызова:
			\inputminted{text}{code/peekResults.txt}
		\end{minipage}
\end{frame}

\subsection{Обобщение}
\begin{frame}
\frametitle{\insertsection} 
\framesubtitle{\insertsubsection}
Для отладочных целей интерфейс Stream определяет метод peek.
С помощью него можем запомнить объекты перед и после вызова не изменив логику.
\begin{align*}
	&.peek(x \rightarrow store(x, time)) \\
	&.call(...)\\
	&.peek(z \rightarrow \{time.increment()\})\\
	&.peek(y \rightarrow store(y, time))
\end{align*}

В результате получим два множества $Before = \{(t_i, x_i)\}, After = \{(t_i, y_i)\}$. Они образуют состояния между вызовами.

Чтобы найти переходы достаточно построить \textbf{отображения}
\begin{align*}
	(t_i, x_i) \rightarrow List[(t_j, y_j)], \ \forall (t_i, x_i) \in Before \\
	(t_i, y_i) \rightarrow List[(t_j, x_j)], \ \forall (t_j, y_j) \in After
\end{align*}

\end{frame}

\subsection{Пример}
\begin{frame}
\frametitle{\insertsection} 
\framesubtitle{\insertsubsection}
Рассмотрим в качестве примера вызов:
\inputminted{java}{code/FlatMapFactorizeExample.java}
\begin{figure}
	\includegraphics<1>[scale=0.8]{img/flatMap/flatMapExample1.png}
	\includegraphics<2>[scale=0.8]{img/flatMap/flatMapExample2.png}
	\includegraphics<3>[scale=0.8]{img/flatMap/flatMapExample3.png}
	\includegraphics<4>[scale=0.8]{img/flatMap/flatMapExample4.png}
	\includegraphics<5>[scale=0.8]{img/flatMap/flatMapExample5.png}
	\includegraphics<6>[scale=0.8]{img/flatMap/flatMapExample6.png}
	\includegraphics<7>[scale=0.8]{img/flatMap/flatMapExample7.png}
	\includegraphics<8>[scale=0.8]{img/flatMap/flatMapExample8.png}
\end{figure}
Решая такие же задачи и для остальных методов, получим для них отображение. Такой подход работает почти для всех методов. Исключение - операции с состоянием. Для них можно придумать отдельное решение.
\end{frame}

\subsection{Исключения}
\begin{frame}
\frametitle{\insertsection} 
\framesubtitle{\insertsubsection}
Подойдет ли такое решение абсолютно для всех вызовов?

Нет, но почти для всех этого достаточно. Проблемы возникают с операциями, обладающими состоянием. Например, для вызова $distinct$ этого недостаточно. В общем случае, обратное отображение восстановить невозможно.

Для $distinct$ обратное отображение можно построить явно при вычислении.
\end{frame}

\subsection{Вычисление}
\begin{frame}
\frametitle{\insertsection} 
\framesubtitle{\insertsubsection}
Выполнение кода трассировки.
\begin{enumerate}
	\item Найти и выбрать использование Stream API вблизи позиции отладчика
	\item Построить выражение с дополнительными вызовами peek и обработкой собранной информации
	\item Скомпилировать новый класс, в котором будет код, вычисляющий выражение.
	\item Загрузить его в виртуальную машину и выполнить.
	\item Получить результат вычислений и интерпретировать его.
\end{enumerate}
\end{frame}

\begin{frame}[noframenumbering]
\frametitle{\insertsection} 
\framesubtitle{\insertsubsection}
Чтобы вычислить выражение для отслеживания исполнения цепочки потоков объектов нужно определить класс, а так же учесть следующие особенности
\begin{itemize}
	\item Поля и методы этого класса.
	
	\item Расположение класса: пакет, объемлющий класс.
	
	\item Доступ к полям и методам объемлющего класса.

	\item Доступ к приватным членам класса из лямбд и анонимных классов.

	\item Минимизация дальнейших обращений к виртуальной машине.
\end{itemize}

\inputminted{java}{code/EvalClass.java}
\end{frame}


\section{Дополнительно}
\begin{frame}
\frametitle{\insertsection} 
\framesubtitle{\insertsubsection}
Описанный подход позволяет исполнять код "в песочнице". Поэтому можем собирать дополнительную информацию:
\begin{itemize}
	\item Информация об исключениях
	\item Промежуточные состояния итераторов
\end{itemize}
\end{frame}

\subsection{Исключения}
\begin{frame}
\frametitle{\insertsection} 
\framesubtitle{\insertsubsection}
Определение вызова и объекта, на котором произошло исключение.
\inputminted{java}{code/Exceptions.java}
\end{frame}


\subsection{Промежуточные состояния итераторов}
\begin{frame}
\frametitle{\insertsection} 
\framesubtitle{\insertsubsection}
\inputminted{java}{code/Spliterator-props.java}
\end{frame}

\begin{frame}
\frametitle{\insertsection} 
\framesubtitle{\insertsubsection}
\inputminted{java}{code/Spliterator-props.java}
Вывод в Java 8: 123

Вывод в Java 9:

Причина - использование характеристик итераторов
\end{frame}


\subsection{Порядок при операциях с состоянием}
\input{frames/extra-stateful-op-order.tex}

\begin{frame}
\frametitle{\insertsection} 
\framesubtitle{\insertsubsection}
\begin{itemize}
	\item Разработан плагин, упрощающий отладку операций над потоками объектов.
	\item Плагин размещен в открытом доступе
	\item Поддержаны все операции стандартной реализации Stream API.
	\item Список поддерживаемых методов можно быстро расширить.
\end{itemize}
\end{frame}
 
\begin{frame}
\frametitle{Что будет сделано}
TODO: Брейкпоинты, evaluation, etc
\end{frame}

\begin{frame}
\frametitle{Оценка полученного решения}
TODO: Трассировка и брейкпоинты, написать нормальный заголовок слайда
\end{frame}
