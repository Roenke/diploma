

\section{Введение}

% Streams
\subsection{Интерфейс потоков объектов}
\begin{frame}
\frametitle{Java Stream API} % Table of contents slide, comment this block out to remove it
% Фича Java 8
% Ленивость
% Функциональный стиль
\begin{itemize}
	\item Stream API - пакет java.util.stream, содержащий набор классов и интерфейсов для упрощения обработки последовательностей элементов.
	\item Появилось в Java 8 
	\item Аналоги в других языках - LINQ в C\#
\end{itemize}
\end{frame}
\subsection{Интерфейс потоков объектов. Реализация}
\begin{frame}
\frametitle{\insertsection} 
\framesubtitle{\insertsubsection}
\begin{itemize}
	\item \textbf{Источник объектов}. Например, массив, коллекция, функция генератор, поток ввода и другие.
	\inputminted{java}{code/Producer.java}
	\item \textbf{Промежуточные операции}. Преобразуют объекты внутри потока.
	\begin{itemize}
		\item Без состояния
		\inputminted{java}{code/Stateless.java}
		\item С состоянием
		\inputminted{java}{code/Stateful.java}
	\end{itemize}
	\item \textbf{Завершающая операция}. Завершает цепочку, преобразуя поток объектов в результат.
	\inputminted{java}{code/Termination.java}
\end{itemize}
\end{frame}

\begin{frame}
\frametitle{\insertsection} 
\framesubtitle{\insertsubsection}
\begin{itemize}
	
	\item Существуют расширения $java.util.stream$ - StreamEx, jOOL. Эти библиотеки совместимы со стандартной реализацией, расширяя стандартные интерфейсы.
\end{itemize}
\end{frame}


% Stream debugging
\section{Отладка потоков объектов}
\subsection{Доступные инструменты}
\begin{frame}
\frametitle{\insertsection} 
\framesubtitle{\insertsubsection}
\begin{itemize}
	\item Точки останова (с условием)
	\item Последовательное исполнение
	\item При помощи метода peek
	\item Автоматическая конвертация в код с обычными управляющими структурами и отладка этого кода
	\item Частичное исполнение и сохранение во временную коллекцию (не во всех средах разработки)
\end{itemize}
\end{frame}

\subsection{Проблемы}
\begin{frame}
\frametitle{\insertsection} 
\framesubtitle{\insertsubsection}
Недостатки отладки интерфейса потоков объекта в сравнении с обычными управляющими структурами
\begin{itemize}
	\item Нетривиальная последовательность исполнения
	\item Отсутствие знаний о промежуточных результатах вычислений 
	\item Отсутствие информации о трансформации объектов
	\item Сложные стеки вызовов
	\item Отладка исключительных ситуаций
\end{itemize}
\end{frame}

\subsection{Пример использования интерфейса потоков в проекте с открытым кодом}
\begin{frame}
\frametitle{\insertsection} 
\framesubtitle{\insertsubsection}
\inputminted{java}{code/Hard.java}
\end{frame}


\section{Цель}
\begin{frame}
	\frametitle{\insertsection} 
	\framesubtitle{\insertsubsection}
	Расширить возможности отладчика для поиска ошибок при использовании интерфейса потоков объектов
\end{frame}

\subsection{Существующие решения}
\begin{frame}
\frametitle{\insertsection} 
\framesubtitle{\insertsubsection}
Существует расширение для отладчика Visual Studio для C\# - OzCode. Одна из его функций - отладчик для LINQ.
\includegraphics[scale=0.35]{img/ozcode.png}
\end{frame}

\subsection{Особенности OzCode}
\begin{frame}[noframenumbering]
\frametitle{\insertsection} 
\framesubtitle{\insertsubsection}
\includegraphics[scale=0.35]{img/ozcode.png}
\end{frame}

\subsection{Подзадачи}
\begin{frame}
\frametitle{\insertsection} 
\framesubtitle{\insertsubsection}
\begin{enumerate}
	\item Распознать вызов Stream API возле текущей позиции отладчика.
	\item Построить промежуточные состояния между вызовами в цепочке.
	\item Построить переходы между состояниями.
	\item Визуализировать результаты.
\end{enumerate}
\end{frame}

\section{Решение}
\section{Поиск вызова}
\begin{frame}[noframenumbering]
\frametitle{\insertsection} 
\framesubtitle{\insertsubsection}
Чтобы распознать вызов будем использовать AST файла.
С учетом следующих особенностей:
\begin{itemize}
	\item Определение типа вызова: 
	\begin{itemize}
		\item Источник - первый вызов (объект), возвращающий наследника Stream<T>.
		\item Промежуточный - вызов на наследнике Stream<T>, возвращающий Stream<T>.
		\item Завершающий - вызов на наследнике Stream<T>, возвращающий что-то отличное от Stream<T>.
	\end{itemize}
	\item Цепочка может не иметь завершающей операции. Такие цепочки нам не интересны (в них нет вычислений)
	\item Может быть несколько подходящих вызовов. Необходимо найти их все.
	\item Цепочка может быть на других уровнях стека вызовов.
	\item Цепочка может быть в объемлющем коде.
\end{itemize}
\end{frame}

\subsection{Обобщение}
\begin{frame}
\frametitle{\insertsection} 
\framesubtitle{\insertsubsection}
Для отладочных целей интерфейс Stream определяет метод peek.
С помощью него можем запомнить объекты перед и после вызова не изменив логику.
\begin{align*}
	&.peek(x \rightarrow store(x, time)) \\
	&.call(...)\\
	&.peek(z \rightarrow \{time.increment()\})\\
	&.peek(y \rightarrow store(y, time))
\end{align*}

В результате получим два множества $Before = \{(t_i, x_i)\}, After = \{(t_i, y_i)\}$. Они образуют состояния между вызовами.

Чтобы найти переходы достаточно построить \textbf{отображения}
\begin{align*}
	(t_i, x_i) \rightarrow List[(t_j, y_j)], \ \forall (t_i, x_i) \in Before \\
	(t_i, y_i) \rightarrow List[(t_j, x_j)], \ \forall (t_j, y_j) \in After
\end{align*}

\end{frame}

\subsection{Пример}
\begin{frame}
\frametitle{\insertsection} 
\framesubtitle{\insertsubsection}
Рассмотрим в качестве примера вызов:
\inputminted{java}{code/FlatMapFactorizeExample.java}
\begin{figure}
	\includegraphics<1>[scale=0.8]{img/flatMap/flatMapExample1.png}
	\includegraphics<2>[scale=0.8]{img/flatMap/flatMapExample2.png}
	\includegraphics<3>[scale=0.8]{img/flatMap/flatMapExample3.png}
	\includegraphics<4>[scale=0.8]{img/flatMap/flatMapExample4.png}
	\includegraphics<5>[scale=0.8]{img/flatMap/flatMapExample5.png}
	\includegraphics<6>[scale=0.8]{img/flatMap/flatMapExample6.png}
	\includegraphics<7>[scale=0.8]{img/flatMap/flatMapExample7.png}
	\includegraphics<8>[scale=0.8]{img/flatMap/flatMapExample8.png}
\end{figure}
Решая такие же задачи и для остальных методов, получим для них отображение. Такой подход работает почти для всех методов. Исключение - операции с состоянием. Для них можно придумать отдельное решение.
\end{frame}

\subsection{Вычисление}
\begin{frame}
\frametitle{\insertsection} 
\framesubtitle{\insertsubsection}
Выполнение кода трассировки.
\begin{enumerate}
	\item Найти и выбрать использование Stream API вблизи позиции отладчика
	\item Построить выражение с дополнительными вызовами peek и обработкой собранной информации
	\item Скомпилировать новый класс, в котором будет код, вычисляющий выражение.
	\item Загрузить его в виртуальную машину и выполнить.
	\item Получить результат вычислений и интерпретировать его.
\end{enumerate}
\end{frame}


\section{Результаты}
\begin{frame}
\frametitle{\insertsection} 
\framesubtitle{\insertsubsection}
\begin{itemize}
	\item Разработан плагин, упрощающий отладку операций над потоками объектов.
	\item Плагин размещен в открытом доступе
	\item Поддержаны все операции стандартной реализации Stream API.
	\item Список поддерживаемых методов можно быстро расширить.
\end{itemize}
\end{frame}
 
\begin{frame}
\frametitle{\insertsection} 
\framesubtitle{\insertsubsection}
\href{https://plugins.jetbrains.com/plugin/9696-java-stream-debugger}{https://plugins.jetbrains.com/plugin/9696-java-stream-debugger}

\animategraphics[autoplay,loop,width=\textwidth]{8}{img/demo/demos-}{0}{188}
\end{frame}


\appendix
\section{Архитектура отладчика}
\begin{frame}
\frametitle{Java Debug Interface} % Table of contents slide, comment this block out to remove it
TODO: JVMTI и это вот всё
\end{frame}
\section{Выражение для вычисления}
\begin{frame}[noframenumbering]
\frametitle{\insertsection} 
\framesubtitle{\insertsubsection}
Чтобы вычислить выражение для отслеживания исполнения цепочки потоков объектов нужно определить класс, а так же учесть следующие особенности
\begin{itemize}
	\item Поля и методы этого класса.
	
	\item Расположение класса: пакет, объемлющий класс.
	
	\item Доступ к полям и методам объемлющего класса.

	\item Доступ к приватным членам класса из лямбд и анонимных классов.

	\item Минимизация дальнейших обращений к виртуальной машине.
\end{itemize}

\inputminted{java}{code/EvalClass.java}
\end{frame}

\section{Гарантии библиотеки}
\begin{frame}[noframenumbering]
\frametitle{\insertsection} 
\framesubtitle{\insertsubsection}
Поток объектов всегда ленивый. Это значит, что объекты не из источника будут браться только когда выполняется терминальная операция и ей нужен объект.
\begin{itemize}
	\item Это исключает ситуации, когда из каких-то промежуточных операций требовались объекты, но затем не использовались
	\item Но это не исключает, что некоторым промежуточным операциям потребуется прочитать более одного объекта, чтобы вернуть один. (см sorted, distinct, и др)
	\item Следствие: нет вызова терминальной операции - нет вычислений
\end{itemize}

\end{frame}
\section{Требования к коду пользователя}
\begin{frame}[noframenumbering]
\frametitle{\insertsection} 
\framesubtitle{\insertsubsection}
"Streams are lazy; computation on the source data is only performed when the terminal operation is initiated, and source elements are consumed only as needed."
\begin{itemize}
	\item Это значит, что объекты могут браться из источника только по необходимости. % ситуация, когда из каких то промежуточных операций забрались объекты, но затем не использовались исключены
	\item Но это не исключает, что некоторым промежуточным операциям потребуется прочитать всё объекты. (см sorted, distinct, и др)
\end{itemize}

\end{frame}
\section{Распознавание вызова}
\begin{frame}[noframenumbering]
\frametitle{\insertsection} 
\framesubtitle{\insertsubsection}
\includegraphics[scale=0.7]{img/ambiguous.png}
\end{frame}


\section{Пример}
\begin{frame}[noframenumbering]
\frametitle{\insertsection} 
\framesubtitle{\insertsubsection}
Рассмотрим пример. Поставим задачу восстановить промежуточные состояния и переходы.
\inputminted{java}{code/StreamWithoutPeeks.java}
\end{frame}

\begin{frame}[noframenumbering]
\frametitle{\insertsection} 
\framesubtitle{\insertsubsection}
Интерфейс Stream определяет для отладочных целей метод peek. Добавим его между всеми вызовами методов Stream<T>.
\inputminted{java}{code/StreamWithPeeks.java}
\end{frame}

\begin{frame}[noframenumbering]
\frametitle{\insertsection} 
\framesubtitle{\insertsubsection}
\fboxsep=0pt
\noindent
	\begin{minipage}[t]{0.48\linewidth}
		Информация, которую можно извлечь:
		\begin{itemize}
			\item Последовательность прохождения объектов через цепочку вызовов
			\item Результат фильтрации
			\item sorted имеет состояние и требует выполнить весь поток до своего вызова
			\item Преобразования объектов
		\end{itemize}
	\end{minipage}
	\hfill%
		\begin{minipage}[t]{0.48\linewidth}
			Результат вызова:
			\inputminted{text}{code/peekResults.txt}
		\end{minipage}
\end{frame}
