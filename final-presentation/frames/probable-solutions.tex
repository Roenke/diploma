\begin{frame}
\frametitle{\insertsection} 
\framesubtitle{\insertsubsection}
Рассмотрим в качестве примера вызов:
\inputminted{java}{code/FlatMapFactorizeExample.java}
\begin{figure}
	\includegraphics<1>[scale=0.8]{img/flatMap/flatMapExample1.png}
	\includegraphics<2>[scale=0.8]{img/flatMap/flatMapExample2.png}
	\includegraphics<3>[scale=0.8]{img/flatMap/flatMapExample3.png}
	\includegraphics<4>[scale=0.8]{img/flatMap/flatMapExample4.png}
	\includegraphics<5>[scale=0.8]{img/flatMap/flatMapExample5.png}
	\includegraphics<6>[scale=0.8]{img/flatMap/flatMapExample6.png}
	\includegraphics<7>[scale=0.8]{img/flatMap/flatMapExample7.png}
	\includegraphics<8>[scale=0.8]{img/flatMap/flatMapExample8.png}
\end{figure}
Решая такие же задачи и для остальных методов, получим для них отображение. Такой подход работает почти для всех методов. Исключение - операции с состоянием. Для них можно придумать отдельное решение.
\end{frame}
